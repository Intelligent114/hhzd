%!TEX program = xelatex
% 完整编译: xelatex -> biber/bibtex -> xelatex -> xelatex
% 欢迎参加《瀚海之巅》项目的供题或征解活动!
% ————————————————————模板使用说明——————————————————————————
%本模板改编自elegantpaper模板,该模板的环境使用方式与大多数模板一致,如有疑问可以阅读官方文件的解释,或直接阅读.cls文件的内容。
%文档标题应当注明征解及其期号和题号,或供题
%作者写你自己即可,但为了排版方便起见,建议在题目末尾“供稿人”处加上你的名字。
%日期随意
%正文部分已经列举了你可能会用到的列表环境的样式,你可以摸索看看怎么合适。
\documentclass[lang=cn,12pt,a4paper]{elegantpaper.cls}

\title{《瀚海之巅》2024 第3期·征解}
\author{中国科大数学科学学院团委\quad《瀚海之巅》项目组}
\date{\zhtoday}

% 本文档命令
\usepackage{array}
\newcommand{\ccr}[1]{\makecell{{\color{#1}\rule{1cm}{1cm}}}}
\usepackage{tikz}   %交换图表
\usepackage{tikz-cd}
\usepackage[mathscr]{euscript}
\usepackage{draftwatermark}         % 所有页加水印
\SetWatermarkText{\includegraphics{image/logo}} % 设置水印内容
\SetWatermarkLightness{0}             % 设置水印透明度 0-1 对图片无效
\SetWatermarkScale{0.6} % 设置水印大小 0-1


\begin{document}

\maketitle
\begin{enumerate}
	\item 
	证明:如果映射 $f:[0,1]\rightarrow[0,1]$ 连续, $f(0)=0,f(1)=1$,并且在 $[0,1]$ 上 $(f\circ f)(x)\equiv x,$ 则 $f(x)\equiv x.$
	
	\begin{flushright}
		\kaishu
		供题人:郭维基
	\end{flushright}
	
	
	\item 称数列$\{a_n\}_{n\geq 0}$为\textbf{完全单调的(completely monotonic)}, 若对任意$k,n\in\mathbb N$, 总有$(-1)^k(\Delta^k a)_n\geq 0$, 这里$\Delta$是差分算子, 即$(\Delta a)_n=a_{n+1}-a_n$.
	
	\begin{enumerate}
	\item 若$\mu$是$[0,1]$上的某个Borel测度, 令$a_n=\int x^n\mathrm{d}\mu$, 验证$\{a_n\}$是完全单调数列.
	
	\item 考察$\{a_n\}$的对数凸性, 即$a_na_{n+2}\geq a_{n+1}^2$, $n\in\mathbb N$. Hausdorff矩问题是说: 对任意的完全单调数列$\{a_n\}$, 是否总存在Borel测度$\mu$满足(a)的条件. 在承认其正确性的前提下, 直接推出$\{a_n\}$具有对数凸性.
	
	\item 考虑更加初等的证明. 设数列$\{a_n\}$是完全单调的.
	
	\begin{enumerate}[i]
	\item 证明: 对任意$k,m,n\in \mathbb N$, \[(-1)^k(\Delta^k a)_m=\sum_{i=0}^{n}(-1)^{k+i}\binom{n}{i}(\Delta^{k+i}a)_{m+n-i}.\]
	
	\item 对于$u,v>0$, $uv=\dfrac{1}{4}$, 证明: $u(\Delta^2 a)_0+(\Delta a)_1+va_2\geq0$.
	
	\item 证明: $a_na_{n+2}\geq a_{n+1}^2$, $n\in\mathbb N$.
\end{enumerate}
		\end{enumerate}
	
	\begin{flushright}
		\kaishu
		供题人:胡洁洋
	\end{flushright}
	
	\item 证明$\zeta(\frac{1}{2}+i \tau) \ll \tau^{\frac{1}{6}}\log \tau \quad (\tau>0, \tau \to \infty).$ 
			\begin{flushright}
			\kaishu
			供题人:杨文颜
		\end{flushright}
		
		\item 计算$j(\sqrt{3} i)$的值.
		\begin{flushright}
			\kaishu
			供题人:杨文颜
		\end{flushright}

\item(Apostol)

我们记
\[G(x)=\sum_{n\geq 1} \dfrac{n^5x^n}{1-x^n},\]
且
\[f(x)=\sum_{n\geq 1, n \textbf{奇数}}\dfrac{n^5x^n}{1+x^n}.\]

\begin{enumerate}
	\item 证明$F(x) = G(x)-34G(x^2)+64G(x^4).$
	
	\item 证明
	
	\[ \sum_{n\geq 1, n \textbf{奇数}} \dfrac{n^5}{1+e^{n\pi}} = \frac{31}{504}.\]
\end{enumerate}
\begin{flushright}
	\kaishu
	供题人:杨文颜
\end{flushright}
	
	
	
	
		\item 对正整数 $n$, 设 $x_{ij} \in [0,1]$, $\forall 1 \leq i,j \leq n$. 证明: \[\prod_{j=1}^{n}(1 - \prod_{i=1}^{n}x_{ij}) + \prod_{i=1}^{n}(1 - \prod_{j=1}^{n}(1 - x_{ij})) \geq 1.\]
		
		
		\begin{flushright}
			\kaishu
			供题人:邓博文
		\end{flushright}
		
		
			\item
			设$L/K$为可分代数扩张. $f\in K[X_1,\dots,X_n]$. 存在$g\in L[X_1,\dots,X_n]$, $f=g^m$. 证明: 存在$c\in K$, $h\in K[X_1,\dots,X_n]$, 使得$f=ch^m$.
			
		
		\begin{flushright}
			\kaishu
			供题人:孙之棋
		\end{flushright}
	

 
 
 \item 设$X_i\sim U(0,1)$独立同分布, $i\geq 1$. 记$S_n=\sum_{i=1}^{n}X_i$, 定义
 \begin{equation*}
 	N=\min\{n\in\mathbb{N}\mid S_n\geq 1\}.
 \end{equation*}
 求$\mathbb{E}[N]$.
 \begin{flushright}
 	\kaishu
 	供题人: 徐思懿\\
 	题源: MSE.
 \end{flushright}
\end{enumerate}


\end{document}
