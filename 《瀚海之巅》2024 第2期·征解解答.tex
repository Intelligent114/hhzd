%!TEX program = xelatex
% 完整编译: xelatex -> biber/bibtex -> xelatex -> xelatex
% 欢迎参加《瀚海之巅》项目的供题或征解活动!
% ————————————————————模板使用说明——————————————————————————
%本模板改编自elegantpaper模板, 该模板的环境使用方式与大多数模板一致, 如有疑问可以阅读官方文件的解释, 或直接阅读.cls文件的内容. 
%文档标题应当注明征解及其期号和题号, 或供题
%作者写你自己即可, 但为了排版方便起见, 建议在题目末尾“供稿人”处加上你的名字. 
%日期随意
%正文部分已经列举了你可能会用到的列表环境的样式, 你可以摸索看看怎么合适. 
\documentclass[lang=cn,12pt,a4paper]{elegantpaper.cls}

\title{《瀚海之巅》2024第2期·征解解答}
\author{中国科大数学科学学院团委\quad《瀚海之巅》项目组}
\date{2024年4月19日}
\usepackage{draftwatermark}         % 所有页加水印
\SetWatermarkText{\includegraphics{image/logo}} % 设置水印内容
\SetWatermarkLightness{0}             % 设置水印透明度 0-1 对图片无效
\SetWatermarkScale{0.6} % 设置水印大小 0-1

\usepackage{array}
\usepackage{tikz}   %交换图表
\usepackage{tikz-cd}
\usepackage{mathtools}
\usepackage{bm}
\newcommand{\set}[1]{\left\{#1\right\}}
\newcommand{\floor}[1]{\left\lfloor#1\right\rfloor}
\newcommand{\norm}[1]{\left\|#1\right\|}
\newcommand{\dif}{\,\mathrm d}
\allowdisplaybreaks[4]
% 本文档命令
\newcommand{\ccr}[1]{\makecell{{\color{#1}\rule{1cm}{1cm}}}}


\usetikzlibrary{backgrounds}
\DeclareMathOperator\Sym{Sym}
\usepackage[mathscr]{euscript}



\begin{document}

\maketitle
\begin{enumerate}

\item 我们来考察有限阿贝尔群的对偶群(\textbf{dual group})的相关性质. 对阶为$n$的有限阿贝尔群$G$, 定义$\hat{G}$是所有$G \to \mathbb C_{\neq 0}$的积性复值函数构成的集合, 也就是说, 对任意$e\in \hat{G}$, 以及任意$a,b\in G$, $e(a\cdot b)=e(a)e(b)$.
\begin{enumerate}

\item $\hat G$是群. 验证$\hat{G}$在以下运算下构成阿贝尔群: \[(e_1\cdot e_2)(a)=e_1(a)e_2(a),\ \forall a\in G.\]

\item 证明: 对任意$e\in \hat{G}$, 以及任意$a\in G$, $e(a)$是某个$n$次单位根.

\item $\hat{G}$的阶. 研究所有$G\to \mathbb{C}$的复值函数构成的向量空间$V$, 在$V$上定义Hermite内积\[(f,g)=\frac{1}{n}\sum_{a\in G}f(a)\overline{g(a)}.\]

\begin{enumerate}[label=(\roman*)]

\item 证明: $\hat{G}$的元素在上述内积下构成一组标准正交向量组, 进而$|\hat{G}|\leq n$.

\item 证明: $\hat{G}$的元素在上述内积下构成一组标准正交基, 进而$|\hat{G}|=n$.
\end{enumerate}

\item $\hat{G}$的结构. 证明: $\hat{G}\cong G$.
\end{enumerate}
\begin{flushright}
	\kaishu
	供题人: 胡洁洋
\end{flushright}
	
	\begin{proof}
		\begin{enumerate}
		\item 对任意$e_1,e_2\in \hat{G}$, 和任意$a,b\in \hat{G}$, \[(e_1\cdot e_2)(a\cdot b)=e_1(a\cdot b)e_2(a\cdot b)=e_1(a)e_2(a)e_1(b)e_2(b)=(e_1\cdot e_2)(a)(e_1\cdot e_2)(b),\]从而$e_1\cdot e_2\in \hat{G}$.
		
		结合律: 任取$e_1,e_2,e_3\in \hat{G}$, 和$a\in G$, 有\[[(e_1\cdot e_2)\cdot e_3](a)=[e_1\cdot (e_2\cdot e_3)](a)=e_1(a)e_2(a)e_3(a),\]从而$(e_1\cdot e_2)\cdot e_3=e_1\cdot (e_2\cdot e_3)$, 结合律成立.
		
		单位元存在性: 令$e_{\mathrm{Id}}=1,\forall a\in G$, 不难验证其为$\hat{G}$的单位元.
		
		逆元存在性: 对$e\in \hat{G}$, 令$e^{-1}(a)=e(a)^{-1}$, 不难验证$e^{-1}\in\hat{G}$, 且$e_{\mathrm{Id}}=e\cdot e^{-1}=e^{-1}\cdot e$.
		
		同时, 任取$e_1,e_2\in \hat{G}$, 类似可验证$e_1\cdot e_2=e_2\cdot e_1$.
		
		综上, $\hat{G}$构成阿贝尔群.
		
		\item 设$1_G$是$G$的单位元. $e(1_G)=e(1_G\cdot 1_G)=e(1_G)^2$, 得$e(1_G)=1$. 又对任意$a\in G$, $a^n=1_G$, 从而\[1=e(1_G)=e(a^{n})=e(a)e(a^{n-1})=\cdots=e(a)^n,\]即$e(a)$是某个$n$次单位根.
		
		\item 
		\begin{enumerate}[label=(\roman*)]
		\item  一方面, 对任意$e\in \hat{G}$, \[(e,e)=\frac{1}{n}\sum_{a\in G}e(a)\overline{e(a)}=\frac{1}{n}\sum_{a\in G}|e(a)|^2=1,\]
		
		另一方面, 对$e_1,e_2\in \hat{G}$, $e_1\neq e_2$, 下证$(e_1,e_2)=0$. 任取$b\in G$满足$e_1(b)\neq e_2(b)$. 
		
		\[\begin{aligned}
			(e_1,e_2)&=\frac{1}{n}\sum_{a\in G}e_1(a)\overline{e_2(a)}\\&=\frac{1}{n}\sum_{a\in G}(e_1\cdot e_2^{-1})(a)\\&=\frac{1}{n}\sum_{a\in G}(e_1\cdot e_2^{-1})(ab\cdot b^{-1})\\&=\frac{(e_1\cdot e_2^{-1})(b)^{-1}}{n}\sum_{a\in G}(e_1\cdot e_2^{-1})(ab)\\&=\frac{(e_1\cdot e_2^{-1})(b)^{-1}}{n}\sum_{a\in G}(e_1\cdot e_2^{-1})(a)\\&=(e_1\cdot e_2^{-1})(b)^{-1}(e_1,e_2),
		\end{aligned}\]由$(e_1\cdot e_2^{-1})(b)\neq 1$, 我们有$(e_1,e_2)=0$. 进而$\hat{G}$的元素在上述内积下构成一组标准正交向量组. 平凡地, $\dim V=n$, 而正交向量组必为线性无关组, 故$|\hat{G}|\leq n$.
		
		\item  我们只需验证任意$f:G\to \mathbb C$都是$\hat{G}$中元素的线性组合. 考虑证明\[f=\frac{n}{|\hat{G}|}\sum_{e\in \hat{G}}(f,e)e.\]
		
		\begin{quote}
			\textbf{引理.} 对任意$a\neq 1_G$, 有\[\sum_{e\in\hat{G}}e(a)=0.\]
		
		\textbf{引理证明:} 先说明存在$e_0\in \hat{G}$, $e_0(a)\neq 1$. 由有限阿贝尔群的结构定理, 设\[G\cong \mathbb{Z}_{p_1^{k_1}}\times \mathbb{Z}_{p_2^{k_2}}\times \cdots\times \mathbb{Z}_{p_l^{k_l}}.\]同构映射$\varphi$将$a$映为$(\overline{\alpha_1},\overline{\alpha_2},\cdots,\overline{\alpha_l})$, 不失一般性, $\overline{\alpha_1}\neq \overline{0}$.取\[e_0(\varphi^{-1}(\overline{\beta_1},\overline{\beta_2},\cdots,\overline{\beta_l}))=\exp\left(\frac{2\beta_1 \pi i}{p_1^{k_1}}\right),\forall \beta_1,\beta_2,\cdots,\beta_l\in \mathbb Z,\]即满足条件. 进而, 类似(i), \[
		\sum_{e\in\hat{G}}e(a)=\sum_{e\in\hat{G}}(e\cdot e_0)(a)=e_0(a)\sum_{e\in\hat{G}}e(a),\]我们有\[\sum_{e\in\hat{G}}e(a)=0.\]
		
		\end{quote}
		
		回原题. 任取$a\in G$, \[\begin{aligned}
			\frac{n}{|\hat{G}|}\sum_{e\in G}(f,e)e(a)&=\frac{1}{|\hat{G}|}\sum_{e\in \hat{G}}e(a)\sum_{b\in G}f(b)\overline{e(b)}\\
			&=\frac{1}{|\hat{G}|}\sum_{b\in G}f(b)\sum_{e\in \hat{G}}e(ab^{-1})\\
			&=\frac{1}{|\hat{G}|}f(a)\sum_{e\in\hat{G}}e(aa^{-1})+\frac{1}{|\hat{G}|}\sum_{\substack{b\in\hat{G}\\b\neq a}}f(b)\sum_{e\in \hat{G}}e(ab^{-1})\\
			&=f(a)+\frac{1}{|\hat{G}|}\sum_{\substack{b\in\hat{G}\\b\neq a}}f(b)\sum_{e\in \hat{G}}e(ab^{-1})\\
			&=f(a).
		\end{aligned}\]故\[f=\frac{n}{|\hat{G}|}\sum_{e\in \hat{G}}(f,e)e.\]
		
		由此, $\hat{G}$的元素在上述内积下构成一组标准正交基, 进而$|\hat{G}|=n$.
	\end{enumerate}
		
		\item 分三步证明.
		
		第一步: 结论对有限循环群成立. 考虑$n$阶循环群$G=\langle a\rangle$. 则$e(a^k)=e(a)^k$, 进而$e$由$e(a)$完全决定, 构造映射$\psi:G\to \hat{G}$, $[\psi (a^k)](a^m)=\exp\left(\dfrac{2km\pi i}{n}\right)$, 不难验证确实是同构映射.
		
		第二步: 证明$\widehat{G_1\times G_2}\cong \hat{G_1}\times \hat{G_2}$. 注意对任意$e\in \hat{G}$, 存在$e_i\in \hat{G_i}(i=1,2)$, 使得对任意$g_i\in G_i(i=1,2)$, $e(g_1,g_2)=(e_1(g_1),e_2(g_2))$, 对任意$e_i\in \hat{G_i}(i=1,2)$, 也有唯一的$e$与它们对应, 进而得证.
		
		第三步: 证明本问题. 注意到\[\begin{aligned}
			\hat{G}&\cong \widehat{\mathbb{Z}_{p_1^{k_1}}\times \mathbb{Z}_{p_2^{k_2}}\times \cdots\times \mathbb{Z}_{p_l^{k_l}}}\\&
			\cong \widehat{\mathbb{Z}_{p_{1}^{k_1}}\times \mathbb{Z}_{p_2^{k_2}}\times \cdots\times \mathbb{Z}_{p_{l-1}^{k_{l-1}}}}\times \widehat{\mathbb{Z}_{p_l^{k_l}}}\\
			&\cdots\\
			&\cong \widehat{\mathbb{Z}_{p_1^{k_1}}}\times \widehat{\mathbb{Z}_{p_2^{k_2}}}\times \cdots\times \widehat{\mathbb{Z}_{p_l^{k_l}}}\\
			&\cong \mathbb{Z}_{p_1^{k_1}}\times \mathbb{Z}_{p_2^{k_2}}\times \cdots\times \mathbb{Z}_{p_l^{k_l}}\\
			&\cong G.
		\end{aligned}\]命题得证.
		\end{enumerate}
		
	\end{proof}
	
	
	\item
(范德蒙德行列式的推广)在平稳随机过程的研究中,出现了下述行列式:
\[
\Delta_n(k_1,x_1;\dots;k_m,x_m)=\begin{vmatrix}
	M_{k_1}^n(x_1)\\
	M_{k_2}^n(x_2)\\
	\vdots\\
	M_{k_m}^n(x_m)
\end{vmatrix},
\]
其中 $x_1,x_2,\dots,x_m$ 是未知量; $k_1,\dots,k_m$ 是正整数, $k_1+k_2+\dots+k_m=n$ ; $M_k^n(x)$ 是 $k\times n$ 阶矩阵,形如
\[
M_k^n(x)=\begin{pmatrix}
	1 & x & x^2 & \cdots & x^{n-1} \\
	0 & 1 & \tbinom{2}{1}x & \cdots & \tbinom{n-1}{1}x^{n-2} \\
	0 & 0 & 1 & \cdots & \tbinom{n-1}{2}x^{n-3} \\
	\vdots & \vdots & \vdots & \ddots & \vdots \\
	0 & 0 & 0 & \dots &\tbinom{n-1}{k-1}x^{n-k}
\end{pmatrix}.
\]
证明:
\[
\Delta_n(k_1,x_1;\dots;k_m,x_m)=\prod\limits_{1\leqslant i<j\leqslant m}(x_j-x_i)^{k_ik_j}.
\]
\begin{flushright}
	\kaishu
	供题人: 郭维基
\end{flushright}
	\begin{proof}
		对 $n$ 做归纳,当 $n=2$ 时,结论显然成立.假设小于 $n$ 时结论均成立,当 $n$ 时,将 $\Delta_n(k_1,x_1;\dots;k_m,x_m)$ 的每一列乘以 $x_1$ ,然后用后一列减去这一乘积.注意到 $\tbinom{n+1}{m+1}-\tbinom{n}{m+1}=\tbinom{n}{m}$ ,故有
		\[
		M_{k_1}^n(x_1)\rightarrow M_{k_1,1}^{n}(x_1)=\begin{pmatrix}
			1 & 0 & 0 & 0 & \cdots & 0 \\
			0 & 1 & x_1 & x_1^2 & \cdots & x_1^{n-2} \\
			0 & 0 & 1 & \tbinom{2}{1}x_1 & \cdots & \tbinom{n-2}{1}x_1^{n-3} \\
			\vdots & \vdots & \vdots & \vdots & \ddots & \vdots \\
			0 & 0 & 0 & \cdots & \cdots & \tbinom{n-2}{k_1-2}x_1^{n-k_1}
		\end{pmatrix}
		\]
		而对于任一 $i=2,3,\dots,m$ 有
		\begin{gather*}
			M_{k_i}^n(x_i)\rightarrow M_{k_i,1}^n(x_i)= \\
			\begin{pmatrix}
				1 & x_i-x_1 & x_i(x_i-x_1) & x_i^2(x_i-x_1) & \cdots & x_i^{n-2}(x_i-x_1) \\
				0 & 1 & (x_i-x_1)+x_i & \tbinom{2}{1}x_i(x_i-x_1)+x_i^2 & \cdots & \tbinom{n-2}{1}x_i^{n-3}(x_i-x_1)+x_i^{n-2} \\
				0 & 0 & 1 & (x_i-x_1)+\tbinom{2}{1}x_i & \cdots & \tbinom{n-2}{2}x_i^{n-4}(x_i-x_1)+\tbinom{n-2}{1}x_i^{n-3} \\
				\vdots & \vdots & \vdots & \vdots & \ddots & \vdots \\
				0 & 0 & 0 & \cdots & \cdots & \tbinom{n-2}{k_i-1}x_i^{n-k_i-1}(x_i-x_1)+\tbinom{n-2}{k_i-2}x_i^{n-k_i}
			\end{pmatrix}
		\end{gather*}
		
		现在将 $\Delta_n$ 按 $M_{k_1,1}^n$ 的第一行展开,则 $M_{k_1,1}(x_1)$ 的剩余部分即为 $M_{k_1-1}^{n-1}(x_1)$ .对于 $M_{k_i}^n(x_i) (i\neq1)$ 的剩余部分,先将第一行公因式的 $(x_i-x_1)$ 提取出来,再用第二行减去第一行,然后第二行也有了公因式 $(x_i-x_1)$ ,将这一公因式提取出来后又用第三行减去第二行, 第三行也有了公因式 $(x_i-x_1)\cdots\cdots$如此反复进行下去,最终会得到 $M_{k_i}^{n-1}(x_i)$ ,并且总共提取出来了 $k_i$ 个 $(x_i-x_1)$ .于是,经过上面的变形,我们得到
		\begin{equation}
			\Delta_n(k_1,x_1;\dots;k_m,x_m)=\prod\limits_{i=2}^m(x_i-x_1)^{k_i}\Delta_{n-1}(k_1-1,x_1;\dots;k_m,x_m)
		\end{equation}
		而由归纳假设,我们有
		\begin{equation}
			\Delta_{n-1}(k_1-1,x_1;\dots;k_m,x_m)=\prod_{2\leqslant i<j\leqslant m}(x_j-x_i)^{k_ik_j}\prod_{t=2}^m(x_t-x_1)^{(k_1-1)k_t}
		\end{equation}
		将(2)代入(1)即得所求证.
	\end{proof}
	
	\item 

\begin{enumerate}[label=(\alph*)]
	\item 证明: 存在无穷多个正整数$n$, 使得$\sigma(n)\varphi(n)$不为完全平方数;
	\item 证明: 存在无穷多个正整数$n$, 使得$\sigma(n)\varphi(n)$为完全平方数.
\end{enumerate}
\begin{flushright}
	\kaishu
	供题人: 邓博文
\end{flushright}
	
		\begin{proof}
		\begin{enumerate}[label=(\alph*)]
			\item 取$n$为素数, 此时$\sigma(n)\varphi(n)=n^2-1$不为完全平方数, 这样的$n$有无穷多个.
			
			\item 设素数从小到大排列为$p_1<p_2<\cdots$.
			若满足$\sigma(n)\varphi(n)$为完全平方数的$n$只有有限个, 取正整数$\alpha$使得这些$n$的最大素因子不超过$p_\alpha$. 由素数定理, 存在充分大的$\varepsilon\in\mathbb{N}^*$, 使得\[\#\left\{p|p\text{是素数}, p\in\left(\dfrac{p_\varepsilon+1}{2},p_\varepsilon\right]\right\}>\alpha.\]
			
			设$\beta \in\mathbb N^*$满足\[p_{\beta}\leq \dfrac{p_\varepsilon+1}{2}<p_{\beta+1},\]则\[\varepsilon-\beta=\pi(p_\varepsilon)-\pi(\dfrac{p_\varepsilon+1}{2})>\alpha,\]即$\varepsilon-\alpha>\beta$. 定义函数$g: \mathbb N^*\mapsto \mathbb N^*$, \[g(t)=\prod_{\substack{p\text{为素数}\\v_p(t)\text{为奇数}}} p.\]注意到\[\prod_{i=\alpha+1}^{\beta}(p_i^2-1)=2^{2\varepsilon-2\alpha}\prod_{i=\alpha+1}^{\varepsilon}\dfrac{p_i-1}{2}\prod_{i=\alpha+1}^{\varepsilon}\dfrac{p_i+1}{2},\]从而\[\Omega\left(\prod_{i=\alpha+1}^{\varepsilon}p_i^2-1\right)\leq \dfrac{p_\varepsilon+1}{2}.\]而对任意集合$A\subseteq \{\alpha+1,\alpha+2,\cdots,\varepsilon\}$,\[g\left(\prod_{i\in A}(p_i^2-1)\right)\in\left\{\prod_{i=1}^{\beta}p_i^{r_i}\Big| r_i\in\{0,1\}, 1\leq i\leq \beta\right\},\]从而$g\left(\prod_{i\in A}(p_i^2-1)\right)$的取值至多有$2^\beta$种可能. 又这样的$A$一共有$2^{\varepsilon-\alpha}$个, 且$2^{\varepsilon-\alpha}>2^\beta$, 故存在两个不同的集合$A_1,A_2\subseteq \{\alpha+1,\alpha+2,\cdots,\varepsilon\}$, 使得\[g\left(\prod_{i\in A_1}(p_i^2-1)\right)=g\left(\prod_{i\in A_2}(p_i^2-1)\right),\]这说明\[\prod_{i\in A_1}(p_i^2-1)\prod_{i\in A_2}(p_i^2-1)\]为完全平方数, 进而\[\prod_{i\in A_1\Delta A_2}(p_i^2-1)\]为完全平方数. 取非空集合$A'=A_1\Delta A_2$, 令\[l=\prod_{i\in A'}p_i,\]则\[\sigma(l)\varphi(l)=\prod_{i\in A_1\Delta A_2}(p_i^2-1)\]为完全平方数. 又$l$含有大于$p_\alpha$的素因子, 这与反证假设矛盾, 原命题得证.
		\end{enumerate}
	\end{proof}
	
	 \item
	 设$P_0(x_0,y_0),P_1(x_1,y_1),\cdots,P_n(x_n,y_n)$是平面直角坐标系$xOy$中的$n+1$个整点,其横坐标满足$x_1-x_0,x_2-x_1,\cdots,x_n-x_{n-1}$是互异的正整数,纵坐标满足$y_0<y_1<\cdots<y_n$,且斜率满足
	 \[\dfrac{y_1-y_0}{x_1-x_0}<\dfrac{y_2-y_1}{x_2-x_1}<\cdots<\dfrac{y_n-y_{n-1}}{x_n-x_{n-1}}.\]
	 已知对$i=0,1,\cdots,n-3$,在直线$P_iP_{i+1},P_{i+1}P_{i+1},P_{i+2}P_{i+3}$所围成三角形的内部与边界上只有两个整点(即$P_{i+1}$与$P_{i+2}$).求证:$x_1-x_0,x_2-x_1,\cdots,x_n-x_{n-1}$至多有$2^{n-1}$种可能的大小顺序.
	 \begin{flushright}
	 	\kaishu
	 	供题人: 徐子健
	 \end{flushright}
	\begin{proof}[证法$1$, 徐子健]
		先证明如下引理.
		\begin{quote}
			\textbf{引理.}
			\[x_2-x_1<\max \{x_1-x_0,x_3-x_2\}.\]
			\textbf{引理证明:}
			不妨设$P_1(0,0)$再设$P_2(p,q)$, 那么我们知道$p,q$互质.
		
		设$pp'\equiv 1 \pmod{q},qq'\equiv 1 \pmod{p}$且$p' \in \{1,2,\cdots,q-1\},q' \in \{1,2,\cdots,p-1\}.$
		
		由$pp'+qq'\equiv 1 \pmod{p},pp'+qq'\equiv 1 \pmod{q}$且$p,q$互质知$pp'+qq'\equiv 1 \pmod{pq}$. 又由于$1<pp'+qq'<2pq$故
		\[pp'+qq'=pq+1.\]
		
		考虑点$X(q',q-p')$于是有
		\[\dfrac{q-p'}{q'}<\dfrac{q}{p}.\]
		
		故$X$在$P_1P_2$下方. 由题设, $X$在$P_0P_1$或者$P_2P_3$下方.
		
		若$X$在$P_0P_1$下方, 设$P_0(-p_1,-q_1)$, 于是
		\[\dfrac{q-p'}{q'}<\dfrac{q}{p_1}<\dfrac{q}{p}.\]
		于是
		\[p_1(q-q')+1\leq q_1q', pq_1\leq p_1q-1.\]
		相乘得
		\[q'(p_1q-1)\geq pp_1(q-p')+p.\]
		于是
		\[p_1=p_1(pp'+qq'-pq)\geq p+q'>p.\]
		此即$x_2-x_1<x_1-x_0$.
		
		若$X$在$P_2P_3$下方, 设$P_3(p+p_2,q+q_2)$, 于是
		\[\dfrac{q}{p}<\dfrac{q_2}{p_2}<\dfrac{p'}{p-q'}.\]
		于是
		\[p_2q+1\leq pq_2, q_2(p-1')\leq p_2p'-1.\]
		相乘得
		\[p(p_2p'-1)\geq(p-q')(p_2q+1).\]
		于是
		\[p_2=p_2(pp'+qq'-pq)\geq 2p-q'>p.\]
		此即$x_2-x_1<x_3-x_2$.
		\end{quote}
		
		引理的叙述对$P_i,P_{i+1},P_{i+2},P_{i+3}$都成立, 反复使用即可得到结论.
	\end{proof}
	\begin{proof}[证法$2$, 杨文颜]
		类似地我们也要证明证法$1$中的引理, 然而为了证明之我们先来证明一个引理.
		
		\begin{figure}[htb!]
			\begin{center}
				\includegraphics[width=0.3\textwidth]{image/tikzit_image0}
			\end{center}
		\end{figure}

		我们构作平行四边形$P_1QP_2Q'$, 由题设平行四边形$P_1QP_2Q'$内仅有$P_1,P_2$为整点.
		\begin{quote}
			\textbf{引理1.} 称坐标为整数或半整数的点为半整点. 若$x_2-x_1>1$且$y_2-y_1>1$, 则平行四边形$P_1QP_2Q'$内仅有$P_1,P,P_2$为半整点.
		
			\textbf{引理1证明.} 仅对$P$的横纵坐标均为半整数的情形证明, 其余两种情形类似.
		
			若不然, 不妨设平行四边形$P_1QP_2Q'$包含如图所示的$G,P,T$三个整点.(包含横向的三个半整点同理会与$y_2-y_1>1$矛盾)
			
			由于$P_0P_1$斜率为正, 且平行四边形$P_1QP_2Q'$不能包含$Z_1$作为整点除非$Z_1=P_1$但这与$x_2-x_1>1$矛盾. 故$P_1$一定位于$GZ_1$这条线下方, 故我们可以看出$GZ_1$和$TZ_2$这一条平行带状区域内只有$Z_2=P_1$是可能的, 但这与$x_2-x_1>1$矛盾. 又由于平行四边形$P_1QP_2Q'$不能包含$Z_2$作为整点, 故$P_1$一定位于$GZ_2$这条线的下方. 重复上述过程, $P_1$无论选取在何处总会得到矛盾. 
			
			\begin{figure}[htb!]
				\begin{center}
					\includegraphics[width=0.45\textwidth]{image/a}
				\end{center}
			\end{figure}
		\end{quote}
		我们回到主要引理\footnote{指证法1中的引理.}的证明.
		
		$x_2-x_1=1$或$y_2-y_1=1$时引理显然成立. 下设$x_2-x_1>1,y_2-y_1>1$. 由引理1, 平行四边形$P_1QP_2Q'$内仅有$P_1,P,P_2$为半整点. 由Minkowski定理可知$S_{P_1QP_2Q'}\leq 1$. 记$Q(x,y)$可以得到
		\[S_{P_1QP_2Q'}=-(y-y_1)(x_2-x)+(y_2-y)(x-x_1)\leq 1.\]
		即
		\[\dfrac{y_2-y}{x_2-x}-\dfrac{y-y_1}{x-x_1}\leq \dfrac{1}{(x_2-x)(x-x_1)} \leq \dfrac{4}{(x_2-x_1)^2}.\]
		记$p_1=y_1-y_0,q_1=x_1-x_0,p_2=y_3-y_2,q_2=x_3-x_2,p=y_2-y_1,q=x_2-x_1$.
		
		于是我们得到
		\[LHS=\dfrac{p_1q_2-p_2q_1}{q_1q_2}\leq\dfrac{4}{q^2}=RHS.\]
		
		若$p_1q_2-p_2q_1\geq 4$, 则$q_1q_2\geq q^2$, 故$\max\{q_1,q_2\}\geq q.$现在只需考虑$p_1q_2-p_2q_1=1,2,3$的情形. 
		\begin{quote}
			\textbf{引理2.} 若$p_1q_2-p_2q_1=n$, 则$nq\geq q_1+q_2$.

			\textbf{引理2证明.} 此时$\dfrac{p_1}{q_1}<\dfrac{p}{q}<\dfrac{p_2}{q_2}$, 故$p_1qq_2<pq_1q_2<p_2q_1q$, 而$pq_1q_2-p_1qq_2=(pq_1-p_1q)q_2\geq q_2, p_2q_1q-pq_1q_2=(p_2q-pq_2)q_1\geq q_1$, 而$p_2q_1q-p_1qq_2=q(p_2q_1-p_1q_2)=n$, 即得.
		\end{quote}
			
		设$p_1q_2-p_2q_1=1$, 由引理2与上述结果$q\geq q_1+q_2$, 而$q_1q_2\geq \dfrac{1}{4}q^2$, 故$4q_1q_2\geq(q_1+q_2)^2$即$(q_1-q_2)^2\leq 0$, 矛盾!
		
		设$p_1q_2-p_2q_1=2$, 由引理2与上述结果$2q\geq q_1+q_2$, 而$q_1q_2\geq \dfrac{1}{2}q^2$, 故$8q_1q_2\geq(q_1+q_2)^2$, 矛盾!
		
		设$p_1q_2-p_2q_1=3$, 由引理2与上述结果$3q\geq q_1+q_2$, 而$q_1q_2\geq \dfrac{3}{4}q^2$, 故$12q_1q_2\geq(q_1+q_2)^2$, 矛盾!
		
		故引理成立.
	\end{proof}
	
	
		\item 证明: 对$n\in\mathbb N^*$, \[\left|\begin{matrix}
		1 & \frac{1}{2} & \frac{1}{3} & \cdots & \frac{1}{n}\\
		\frac{1}{2} & 1 & \frac{1}{2} & \cdots & \frac{1}{n-1}\\
		\frac{1}{3} & \frac{1}{2} & 1 & \cdots & \frac{1}{n-2}\\
		\vdots & \vdots & \vdots & \ddots & \vdots\\
		\frac{1}{n} & \frac{1}{n-1} & \frac{1}{n-2} & \cdots & 1\\
	\end{matrix}\right|>0.\] 
	
	\begin{flushright}
		\kaishu
		供题人: 邓博文
	\end{flushright}
	
	\begin{proof}
		命题等价于\[\sum_{i=1}^{n}\sum_{j=1}^{n}\dfrac{1}{1+|i-j|}x_ix_j\]为正定二次型, 也即\[\sum_{i=1}^{n}\sum_{j=1}^{n}\dfrac{a_ia_j}{1+|i-j|}\geq 0,\]等号成立当且仅当$a_1=a_2=\cdots=a_n=0$.
		
		\begin{quote}
			\textbf{断言1.} \[\sum_{k\in\mathbb Z}\mathrm e^{ikx}r^{|k|}=\dfrac{1-r^2}{1-2r\cos x+r^2}.(r<1)\]
		
		\textbf{断言1证明.} \[\begin{aligned}
			LHS&=-1+\sum_{k=0}^{+\infty}\mathrm e^{ikx}r^{k}+\sum_{k=0}^{+\infty}\mathrm e^{-ikx}r^{k}\\&=-1+\dfrac{1}{1-r\mathrm e^{ix}}+\dfrac{1}{1-r\mathrm e^{-ix}}\\&=RHS.
		\end{aligned}\]
		
		\textbf{断言2.} \[\sum_{k\in\mathbb Z}\dfrac{1}{1+|k|}\mathrm e^{ikx}=\int_{0}^{1}\dfrac{1-r^2}{1-2r\cos x+r^2}\mathrm dr>0.\]
		
		\textbf{断言2的证明.} \[\begin{aligned}
			LHS&=\sum_{k\in\mathbb Z}\int_{0}^{1}r^{|k|}\mathrm dr\cdot\mathrm e^{ikx}\\
			&=\int_{0}^{1}\sum_{k\in\mathbb Z}r^{|k|}\mathrm e^{ikx}\mathrm dr\\
			&=\int_{0}^{1}\sum_{k\in\mathbb Z}\mathrm e^{ikx}r^{|k|}\mathrm dr\\
			&=\int_{0}^{1}\dfrac{1-r^2}{1-2r\cos x+r^2}\mathrm dr\\
			&>0.
		\end{aligned}\]
		
		\textbf{断言3.} 对$k\in\mathbb Z_{\neq 0}$, \[\int_{0}^{2\pi}\mathrm e^{ikx}\mathrm dx=0.\]此断言平凡地成立.
		
		\end{quote}
		
		从而结合断言3, \[
		\begin{aligned}
			&\int_{0}^{2\pi}\left(\sum_{k\in\mathbb Z}\dfrac{1}{1+|k|}\mathrm e^{ikx}\right)\left|\sum_{k=1}^{n}a_k\mathrm e^{ikx}\right|^2\mathrm dx\\
			=&\int_{0}^{2\pi}\left(\sum_{k\in\mathbb Z}\dfrac{1}{1+|k|}\mathrm e^{ikx}\right)\left(\sum_{k=1}^{n}a_k\mathrm e^{ikx}\right)\left(\sum_{k=1}^{n}a_k\mathrm e^{-ikx}\right)\mathrm dx\\
			=&\int_{0}^{2\pi}\left(\sum_{k\in\mathbb Z}\dfrac{1}{1+|k|}\mathrm e^{ikx}\right)\left(\sum_{1\leq s,t\leq n}a_sa_t\mathrm e^{i(s-t)x}\right)\mathrm dx\\
			=&\int_{0}^{2\pi}\sum_{1\leq s,t\leq n}a_sa_t\mathrm e^{i(t-s)x}\dfrac{1}{1+|s-t|}\mathrm e^{i(s-t)}\mathrm dx\\
			=&2\pi\sum_{1\leq s,t\leq n}\dfrac{a_sa_t}{1+|s-t|},
		\end{aligned}
		\]
		
		则由断言2,\[\sum_{1\leq i,j\leq n}\dfrac{a_ia_j}{1+|i-j|}=\dfrac{1}{2\pi}\int_{0}^{2\pi}\left(\sum_{k\in\mathbb Z}\dfrac{1}{1+|k|}\mathrm e^{ikx}\right)\left|\sum_{k=1}^{n}a_k\mathrm e^{ikx}\right|^2\mathrm dx\geq 0.\]
		
		取等时, $\forall x\in (0,2\pi)$, \[\sum_{k=1}^{n}a_k\mathrm e^{ikx}=0.\] 	
		
		故对任意$x_1,x_2,\cdots,x_n\in(0,2\pi)$, 有\[\left\{\begin{aligned}
			&\mathrm e^{ix_1}a_1+\mathrm e^{2ix_1}a_2+\cdots+\mathrm e^{nix_1}a_n=0,\\
			&\mathrm e^{ix_2}a_1+\mathrm e^{2ix_2}a_2+\cdots+\mathrm e^{nix_2}a_n=0,\\
			&\qquad\cdots\\
			&\mathrm e^{ix_n}a_1+\mathrm e^{2ix_n}a_2+\cdots+\mathrm e^{nix_n}a_n=0,\\
		\end{aligned}\right.\]
		
		系数行列式\[\left|\begin{matrix}
			\mathrm e^{ix_1}& \mathrm e^{2ix_1} & \cdots & \mathrm e^{nix_1}\\
			\vdots & \vdots &  & \vdots\\
			\mathrm e^{ix_n} & \mathrm e^{2ix_n} & \cdots & \mathrm e^{nix_n}\\
		\end{matrix}\right|=\prod_{j=1}^{n}\mathrm e^{ix_j}\prod_{1\leq s<t\leq n}(e^{ix_t}-e^{ix_s}),\]取定一组$(x_1,\cdots,x_n)$, 使得$\prod\limits_{1\leq s<t\leq n}(e^{ix_t}-e^{ix_s})\neq 0$, 这表明$a_1=a_2=\cdots=a_n=0$.
		
		进而, 原命题成立.
		
		
	\end{proof}
	
	
		\item  设$\mathscr{A}\subset{\mathscr{P}}(X)$是一个代数, $\mathscr{A}_\sigma$是$\mathscr{A}$中集合的可数并的全体, 
	$\mathscr{A}_{\sigma\delta}$是$\mathscr{A}_\sigma$中集合的可数交的全体.设$\mu$是$\mathscr{A}$上的一个预测度, $\mu^{\star}(E)=$inf$\left\{\sum\limits_{i=1}^{\infty}\mu(E_i):E_i\in{\mathscr{A}},E\subset{\bigcup\limits_{i=1}^{\infty}E_i}\right\}$是$\mu$诱导的外测度, 证明:  \begin{enumerate}[(1)]
		\item 对于任意$E\in{X}$以及$\epsilon>0$, 存在$A\in{\mathscr{A}_{\sigma}}$满足$E\subset{A}$且$\mu^{\star}(A)\leq\mu^{\star}(E)+\epsilon$.
		\item 如果$\mu^{\star}(E)<\infty$, 则$E$是$\mu^{\star}$-可测的当且仅当存在$B\in{\mathscr{A_{\sigma\delta}}}$满足$E\subset{B}$且$\mu^{\star}(B\setminus{E})=0$.
		\item 设$(X,M,\mu)$是一个测度空间, $\mu^{\star}(E)=$inf$\{\sum\limits_{i=1}^{\infty}\mu(E_i):E_i\in{M},E\subset{\bigcup\limits_{i=1}^{\infty}E_i}\}$是$\mu$诱导的外测度, $M^{\star}$是$\mu^{\star}$-可测集全体, $\widehat{\mu}=\mu^{\star}|_{M^{\star}}$.证明: $\widehat{\mu}$是$\mu$的完备化的饱和化.
	\end{enumerate}
	\begin{flushright}
		\kaishu
		供题人: 李梦喆
	\end{flushright}
	 \begin{proof}
		\begin{enumerate}[(1)]
			\item 根据$\mu^{\star}$定义知, $\forall\epsilon>0$, $\exists E_i\in{\mathscr{A}}$, 使得$E\subset{\bigcup\limits_{i=1}^{\infty}E_i}$且$\sum\limits_{i=1}^{\infty}\mu(E_i)\leq\mu^{\star}(E)+\epsilon$, 令$A=\bigcup\limits_{i=1}^{\infty}E_i$, 则$A\in{\mathscr{A}_\sigma}$满足$E\subset{A}$, $\mu^{\star}(A)=\mu^{\star}(\sum\limits_{i=1}^{\infty }E_i)\leq\sum\limits_{i=1}^{\infty}\mu^{\star}(E_i)\leq\mu^{\star}(E)+\epsilon$.
			\item ($\Longrightarrow$): 根据$(1)$, 存在$A_n\in{\mathscr{A}}$, 满足$E\subset{A_n}$且$\mu^{\star}(A_n)\leq\mu^{\star}(E)+\frac{1}{n}$.因为$E$是$\mu^{\star}-$可测的, $\mu^{\star}(A_n)=\mu^{\star}(A_n\cap{E})+\mu^{\star}(A_n\cap{E^{c}})=\mu^{\star}(E)+\mu^{\star}(A_n\setminus{E})$.因为$\mu^{\star}(E)<{\infty}$, 两边同时减去$\mu^{\star}(E)$得到$\mu^{\star}(A_n\setminus{E})=\mu^{\star}(A_n)-\mu^{\star}(E)$.令$B=\bigcap\limits_{n=1}^{\infty}A_n$, 则$E\subset{B}$, $0\leq\mu^{\star}(B\setminus{E})\leq\mu^{\star}(A_n\setminus{E})=\mu^{\star}(A_n)-\mu^{\star}(E)\leq\frac{1}{n}$, 令$n\rightarrow\infty$得$\mu^{\star}(B\setminus{E})=0$.\\($\Longleftarrow$): 根据定义, 我们只要证明对于任意的$A\subset{X}$, $\mu^{\star}(A)\geqslant\mu^{\star}(A\cap{E})+\mu^{\star}(A\cap{E^{c}})$.由$Carath\Acute{e}odory$定理知$B$是$\mu^{\star}-$可测的, 因此$\mu^{\star}(A)=\mu^{\star}(A\cap{B})+\mu^{\star}(A\cap{B^{c}})$.所以我们只需证明$\mu^{\star}(A\cap{B})+\mu^{\star}(A\cap{B^{c}})\geqslant\mu^{\star}(A\cap{E})+\mu^{\star}(A\cap{E^{c}})$.因为$A\cap{B}\supset{A\cap{E}}$, 因此$\mu^{\star}(A\cap{B})\geqslant\mu^{\star}(A\cap{E})$.由于$\mu^{\star}(A\cap{E})\leq\mu^{\star}(E)<\infty$, 因此我们只需证明$\mu^{\star}(A\cap{B^{c}})\geqslant\mu^{\star}(A\cap{E^{c}})$.根据条件知$\mu^{\star}(B\cap{E^{c}})=\mu^{\star}(B\setminus{E})=0$.注意到$A\cap{E^{c}}\subseteq{(A\cap{B^{c}})\cup(B\cap{E^{c}})}$, 因此$\mu^{\star}(A\cap{B^{c}})=\mu^{\star}(A\cap{B^{c}})+\mu^{\star}(B\cap{E^{c}})\geqslant\mu^{\star}(A\cap{E^{c}})$.
			\item 
			我们记$M$的完备化为$\overline{M}$, $M$的饱和化为$\widetilde{M}$.我们先证明以下三个引理
			
			\begin{quote}
				\textbf{引理1.} 如果$E\in{M^{\star}}$且$\mu^{\star}(E)<\infty$,则$E\in{\overline{M}}$.
				
				\textbf{引理1证明.} 因为$E\in{M^{\star}}$, 所以$E$和$E^{c}$都是$\mu^{\star}-$可测的. 根据$(2)$, 存在$B_1$、$B_2\in{M}$, 使得$E\subset{B_1},\mu^{\star}(B_1\setminus{E})=0,E^{c}\subset{B_2},\mu^{\star}(B_2\setminus{E})=0$.因此$E=B_2^{c}\cup({E\setminus{B_2^{c}}})$, $B_2^{c}\in{M}$, $E\setminus{B_1^{c}}\subset{B_1\setminus{B_1^{c}}}=B_1\cap{B_2}$, 由完备化的定义知我们只需证明$\mu^{\star}(B_1\cap{B_2})=0$.由于$B_1\cap{B_2}\subset{(B_1\cap{E^{c}})\cup({B_2\cap{E}})}$, 因此$0\leq\mu^{\star}(B_1\cap{B_2})\leq\mu^{\star}(B_1\cap{E^{c}})+\mu^{\star}({B_2\cap{E}})=0$, 这就推出$\mu^{\star}(B_1\cap{B_2})=0$.
				
				\textbf{引理2.} $\forall{E}\subset{X}$, $\exists{B\in{M}}$, 满足$E\subset{B}$且$\mu^{\star}(B)=\mu^{\star}(E)$.
				
				\textbf{引理2证明.} 根据$(1)$, 存在$A_n\in{M}$, 满足$E\subset{A_n}$且$\mu^{\star}(A_n)\leq\mu^{\star}(E)+\frac{1}{n}$.令$B=\bigcap\limits_{n=1}^{\infty}A_n$, 则$E\subset{B}$且$\mu^{\star}(E)\leq\mu^{\star}(B)\leq\mu^{\star}(E)+\frac{1}{n}$, 令$n\rightarrow\infty$, 即得$\mu^{\star}(B)=\mu^{\star}(E)$.
				
				\textbf{引理3.} 如果$A\in\overline{M}$, 则$A\in{M^{\star}}$且$\mu^{\star}(A)=\overline{\mu}(A)$.
				
				\textbf{引理3证明.} 根据完备化的定义, 设$A=E\cup{F}$, 其中$E\in{M}$, $F\subset{N}$, $\mu(N)=0$.因此$\mu^{\star}(F)\leq\mu^{\star}(N)=\mu(N)=0\Longrightarrow\mu^{\star}(F)=0$.$\forall{P\subset{X}}$, $\mu^{\star}(P\cap{F})\leq\mu^{\star}(F)=0$, 因此$\mu^{\star}(P)\geqslant\mu^{\star}(P\cap{F^{c}})=\mu^{\star}(P\cap{F^{c}})+\mu^{\star}(P\cap{F})$, 所以$F$是$\mu^{\star}-$可测的.因为$E\in{M}\subset{M^{\star}}$, 所以$E$也是$\mu^{\star}-$可测的$\Longrightarrow{A=E\cup{F}}$是$\mu^{\star}-$可测的, 即$A\in{M^{\star}}$.根据定义, ${\overline\mu}(A)=\mu(E)=\mu^{\star}(E)$.而$\mu^{\star}(E)\leq\mu^{\star}(A)=\mu^{\star}(E\cup{F})\leq\mu^{\star}(E)+\mu^{\star}(F)=\mu^{\star}(E)$, 因此$\mu^{\star}(A)=\mu^{\star}(E)=\overline{\mu}(A)$.
			\end{quote}

			回到原题, 我们首先证明$\widetilde{\overline{M}}=M^{\star}$.任取$E\in{M^{\star}}$, 对于$\forall{A}\in{\overline{M}}$, 其中$A$满足$\overline{\mu}(A)<\infty$, 由引理3知$A\in{M^{\star}}$且$\mu^{\star}(A)=\overline{\mu}(A)$, 因此$E\cap{A}\in{M^{\star}}$且$\mu^{\star}(E\cap{A})\leq\mu^{\star}(A)<\infty$, 再由引理1知$E\cap{A}\in{\overline{M}}$.因此$E\in{\widetilde{\overline{M}}}$, 故$M^{\star}\subset{\widetilde{\overline{M}}}$.反过来, 任取$E\in{\widetilde{\overline{M}}}$, 我们只需证明对于$\forall{A\subset{X}}$, $\mu^{\star}(A)\geqslant\mu^{\star}(A\cap{E})+\mu^{\star}(A\cap{E^{c}})$.由引理2知存在$B\in{M}$, 满足$A\subset{B}$且$\mu^{\star}(B)=\mu^{\star}(A)$, 由外测度的单调性知我们只需证明$\mu^{\star}(B)\geqslant\mu^{\star}(B\cap{E})+\mu^{\star}(B\cap{E^{c}})$.若$\mu^{\star}(B)=\infty$, 则不等式自然成立, 下讨论$\overline{\mu}(B)=\mu(B)=\mu^{\star}(B)<\infty$的情况.此时由饱和化的定义知因为$E$、$E^{c}\in{\widetilde{\overline{M}}}$, 所以$E\cap{B}$、$E^{c}\cap{B}\in{\overline{M}}$, 由引理3和测度的可加性知$\mu^{\star}(B)=\overline{\mu}(B)=\overline{\mu}(B\cap{E})+\overline{\mu}(B\cap{E^{c}})=\mu^{\star}(B\cap{E})+\mu^{\star}(B\cap{E^{c}})$.因此, $\widetilde{\overline{M}}\subset{M^{\star}}$, 综上我们便证明了$\widetilde{\overline{M}}=M^{\star}$.以下证明$\widetilde{\overline{\mu}}=\mu^{\star}$.任取$E\in{M^{\star}}$, 若$E\in{\overline{M}}$, 由饱和测度定义知$\widetilde{\overline{\mu}}(E)=\overline{\mu}(E)=\mu^{\star}(E)$;若$E\notin{\overline{M}}$, 则由引理1知$\mu^{\star}(E)=\infty$, 由饱和测度定义知$\widetilde{\overline{\mu}}(E)=\infty=\mu^{\star}(E)$.
		\end{enumerate}
		
		
		
	\end{proof}
	
	\item 
设$A(t)$, $B(t)$和$C(t)$是三个实值连续可微函数, 满足方程组:
$$
\begin{cases}
	A'=4\dfrac{(B-C)^2-A^2}{BC}\\
	B'=4\dfrac{(A-C)^2-B^2}{AC}\\
	C'=4\dfrac{(A-B)^2-C^2}{AB}.
\end{cases}
$$
在给定初值条件$A(0)=A_0>0, B(0)=B_0>0, C(0)=C_0>0$的情况下, 求证:\\
(1) 该方程组的解的右行最大存在区间有限.\\
(2) 设这个最大的右行区间为$[0,T)$, 则$\lim\limits_{t\to T}A(t)=\lim\limits_{t\to T}B(t)=\lim\limits_{t\to T}C(t)=0$, 且$\lim\limits_{t\to T}\frac{A(t)}{B(t)}=\lim\limits_{t\to T}\frac{B(t)}{C(t)}=1$.
\begin{flushright}
	\kaishu
	供题人: 计科羽
\end{flushright}
\begin{proof}
由对称性, 我们可以假设初值的大小关系$A_0\leq B_0\leq C_0$, 然后证明$A(t)\leq B(t)\leq C(t)$对任意$t\in [0,T)$成立:
\[(B-A)'=4\frac{(C-A)^2B-B^3-(B-C)^2A+A^3}{ABC}=4(B-A)\frac{C^2-(A+B)^2}{ABC}.\]
由于$\frac{C^2-(A+B)^2}{ABC}$为$[0,T)$光滑函数, 且$B_0-A_0\geq 0$, 可知$B(t)-A(t)\geq 0$对$\forall t\in[0,T)$成立, 同理$C(t)\geq B(t)$对$\forall t\in[0,T)$成立.

现在做$(1)$, 为此研究$C$的变化: 
\[C'=-8+\frac{4(A^2+B^2-C^2)}{AB}\leq -8+\frac{4A}{B}\leq -4.\]
这表明$T<+\infty$, 否则$C$会在有限时间内变为$0$, 这是不允许的.

下面对$C$的上界进行控制, 为此, 计算:
\[\left(\frac{C-A}{A}\right)'=\frac{C'A-A'C}{A^2}=8\left(\frac{C-A}{A}\right)\left(\frac{B-A-C}{AB}\right)\leq 0.\]
于是$\frac{C-A}{A}\leq \frac{C_0-A_0}{A_0}$, 由此可得$C(t)\leq \lambda A(t),\forall t\in [0,T)$, 这表明
\[\lim_{t\rightarrow T^- }A(t)=\lim_{t\rightarrow T^- }B(t)=\lim_{t\rightarrow T^- }C(t)=0.\]

最后来证$(2)$, 为此只需要证$\displaystyle\lim_{t\rightarrow T^-}\frac{C(t)-A(t)}{A(t)}=0$. 假设不然, 则\[\left|\int_{0}^T\frac{\mathrm{d}}{\mathrm{dt}}\left[\log\left(\frac{C-A}{A}\right)\right] \mathrm{d}t\right|<+\infty.\]
根据前面的计算, 这表明
\[\left|\int_0^T\frac{B-A-C}{AB}\mathrm{d}t\right|<+\infty.\]
由于$-\frac{C}{AB}\leq \frac{B-A-C}{AB}\leq -\frac{1}{B}$, 可知$\frac{B-A-C}{AB}\sim -\frac{1}{A}(t\rightarrow T^-)$, 于是只需研究$A^{-1}$作为$[0,T]$上瑕积分的敛散性.
\[A'=-8+4\frac{B^2+C^2-A^2}{BC}\geq -8+4\frac{C}{B}\geq -4\,\lim_{t\rightarrow T^-}A(t)=0.\]
得到$A(t)\leq 4(T-t)$, 故$A^{-1}\geq \frac{1}{4(T-t)}$, 它在$[0,T]$上的积分发散! 这样便得到矛盾. 由此
\[\lim_{t\rightarrow T^-}\frac{C(t)}{A(t)}=1\Rightarrow \lim_{t\rightarrow T^-}\frac{B(t)}{C(t)}=\lim_{t\rightarrow T^-}\frac{A(t)}{B(t)}=1.\]

\begin{remark}
此题背景是$\rm{Ricci}$流理论中的$\rm{Isenberg}$-$\rm{Jackson}$定理, 其具体研究了在对称性充分好的$\mathbb{S}^3$上, 给定初始度量后相应$\rm{Ricci}$流的解的性态. 为此, 视$\mathbb{S}^3=SU(2)$为李群并考虑其李代数生成元
\[x_1=\begin{pmatrix}
 i & 0\\
 0 & -i\\
\end{pmatrix},x_2=\begin{pmatrix}
0 & 1\\
-1 & 0\\
\end{pmatrix},x_3=\begin{pmatrix}
0 &i\\
i & 0\\
\end{pmatrix}.
\]
之后取$\{x_1,x_2,x_3\}$的对偶左不变$1$-形式$\{\sigma^1,\sigma^2,\sigma^3\}$. 则$\mathbb{S}^3$上的度量可以表示为
\[g=A\sigma^1\otimes \sigma^1+B\sigma^2\otimes \sigma^2+C\sigma^3\otimes \sigma^3.\]
将其代入$\rm{Ricci}$流方程$\frac{\partial g(t)}{\partial t}=-2\rm{Ric}(g(t))$中就可计算得到题目中的方程.

在该题目结论的基础上, 还可进一步得到
\[\lim_{t\rightarrow T^-}\frac{A(t)}{4(T-t)}=\lim_{t\rightarrow T^-}\frac{B(t)}{4(T-t)}=\lim_{t\rightarrow T^-}\frac{C(t)}{4(T-t)}=1.\]
这表明度量$\frac{1}{4(T-t)}g(t)$在$t\rightarrow T^-$时收敛于$\mathbb{S}^3$上标准度量, 这就是$\rm{Ricci}$流研究中常用的爆破技巧.
\end{remark}

\end{proof}
	
	
		\item $A$是一个戴德金整环, $K$为其分式域, $L|K$为一个可分扩张,$N$为$L$的伽罗瓦闭包, $\mathfrak{p}$为$A$的素理想, 证明: $\mathfrak{p}$在$L$上完全分裂当且仅当$\mathfrak{p}$在$N$上完全分裂.
	\begin{flushright}
		\kaishu
		供题人: 励随之
	\end{flushright}
	\begin{proof}
		记$A$在$L$中的整闭包为$B$,在$N$中的整闭包为$C$,熟知$B,C$都是戴德金整环,考察$\mathfrak{p}$在其中的分解情况
		\[\mathfrak{p}=\prod_{i=1}^n \mathfrak{p_i}=\prod_{i=1}^n \prod_{j=1}^{m_i} \mathfrak{q_{i,j}}.\]
		其中$\mathfrak{p_i}$是$B$的素理想,$\mathfrak{q_{i,j}}$是$C$的素理想,$\mathfrak{p_i}=\prod_{j=1}^{m_i} \mathfrak{q_{i,j}}$.
		
		假设$\mathfrak{p}$在$N$上完全分裂,则$\mathfrak{q_{i,j}}$两两不同,$[C/\mathfrak{q_{i,j}}:A/\mathfrak{p}]=1$.由于理想的唯一分解性,这当然给出$\mathfrak{p_i}$是$B$中互不相同的素理想,并且$B/\mathfrak{p_i}$作为$C/\mathfrak{q_{i,j}}$的子域,自然有$\mathfrak{p_i}$在$\mathfrak{p}$上惯性度数为$1$,即$\mathfrak{p}$在$L$上完全分裂.
		
		反之设$\mathfrak{p}$在$L$上完全分裂.记$G=\mathrm{Gal}(N|K),H=Gal(N|L),P_{\mathfrak{p}}$为所有在$\mathfrak{p}$上的$B$的素理想构成的集合,$\mathfrak{P}$为某个在$\mathfrak{p}$上的$C$中素理想,$G_{\mathfrak{P}}=\{\sigma\in G|\sigma \mathfrak{P}=\mathfrak{P}\}$,$H\backslash G/G_{\mathfrak{P}}$为$G$关于$H$和$G_{\mathfrak{P}}$的双陪集,熟知如下双射:
		\[H\backslash G/G_{\mathfrak{P}}\rightarrow P_{\mathfrak{p}}:H\sigma G_{\mathfrak{P}}\rightarrow \sigma{\mathfrak{P}}\cap L.\]
		$\mathfrak{p}$在$L$中完全分裂即$|P_{\mathfrak{p}}|=[L:K]=[G:H]$,即双陪集的个数与$H$的右陪集个数一致,而双陪集为有限个右陪集的不交并,因此有$H\sigma G_{\mathfrak{P}}= H\sigma,\forall \sigma\in G$,这就给出$G_{\mathfrak{P}}$的所有共轭都在$H$中,记$G_{\mathfrak{P}}$生成的正规子群为$N(G_{\mathfrak{P}})$,那么$N(G_{\mathfrak{P}})\leq H$,根据伽罗瓦对应,其对应于比$L$大的$K$的某个正规扩张,由于$N$是正规闭包,因此就是$N$,这说明$N(G_{\mathfrak{P}})=1$,给出$G_{\mathfrak{P}}=1$,而在$\mathfrak{p}$上的$C$中不同的素理想和$G/G_{\mathfrak{P}}$一一对应,因此在$C$中共有$|\mathrm{Gal}(N|K)|=|N:K|$个素理想在$\mathfrak{p}$上,根据基本等式即知$\mathfrak{p}$在$N$上完全分裂.
		
	\end{proof}
	
	 \item
	我们将构造两个含幺交换环$R,S$,使得$R,S$不同构, 但多项式环$R[T]$与$S[T]$同构.
	\begin{enumerate}
		
		\item 前置: 对称代数(Symmetric Algebra)
		\begin{enumerate}
			\item 定义: 设$A$为含幺交换环, $M$为$A-$模,$S(M)$是一个$A-$代数, 使得对所有$A-$代数$D$,以及$A-$模同态$M\rightarrow D$,存在唯一$A$代数同态$S(M)\rightarrow D$使得以下图表交换.
			% https://q.uiver.app/#q=WzAsMyxbMCwwLCJNIl0sWzIsMCwiUyhNKSJdLFsyLDIsIkQiXSxbMCwxXSxbMCwyXSxbMSwyLCIiLDAseyJzdHlsZSI6eyJib2R5Ijp7Im5hbWUiOiJkb3R0ZWQifX19XV0=
			\[\begin{tikzcd}
				M && {S(M)} \\
				\\
				&& D
				\arrow[from=1-1, to=1-3]
				\arrow[from=1-1, to=3-3]
				\arrow[dotted, from=1-3, to=3-3]
			\end{tikzcd}\]
			以下证明几个对称代数的简单性质:
			\item $S(M)$具有唯一性.
			\item 对自由模$M=\oplus_{i=1}^n AX_i$, $S(M)=A[X_1,\dots,X_n]$;对$M=\oplus_{i=1}^n AX_i/(Ay_1+\dots+Ay_m)$, $S(M)=A[X_1,\dots,X_n]/(y_1+\dots+y_m).$
			\item $S(M\oplus N)=S(M)\otimes_A S(N).$
		\end{enumerate}
		\item 令$A=\mathbb{R}[X,Y,Z]/(X^2+Y^2+Z^2-1)=\mathbb{R}[x,y,z]$, 其中$x,y,z$分别为$X,Y,Z$在商环中的像.
		\begin{enumerate}
			\item 令
			\begin{align*}
				\phi:A^3&\rightarrow A \\ (a,b,c)&\longmapsto ax+by=cz
			\end{align*}
			令$E=\mathrm{ker}\phi$, 则$A^3\stackrel{\thicksim }{\rightarrow}E\oplus A$(提示:用分裂引理(splitting lemma)).
			\item 证明$S(E)[T]\stackrel{\thicksim }{\rightarrow}A[P,Q,T]$,其中$P,Q,T$为多项式环的变元.
			\item 证明$S(E)=A[U,V,W]/(xU+yV+zW)$.
			\item 证明$S(E)$与$A[P,Q]$不同构.
		\end{enumerate}
		\begin{flushright}
			供题人: 孙之棋
		\end{flushright}
		
		
	
\end{enumerate}
\begin{proof}
	\begin{enumerate}
		\item ii:泛性质具有唯一性.\\
		iii:略\\
		iv:由交换图知.% https://q.uiver.app/#q=WzAsOSxbMCwwLCJNIl0sWzAsMV0sWzAsMiwiTiJdLFsxLDEsIk1cXG9wbHVzIE4iXSxbMiwwLCJTKE0pIl0sWzIsMiwiUyhOKSJdLFszLDFdLFs0LDEsIkQiXSxbMiwxLCJTKE0pXFxvdGltZXNfQSBTKE4pIl0sWzAsN10sWzIsN10sWzAsM10sWzIsM10sWzQsOF0sWzUsOF0sWzQsNywiXFxleGlzdHMgISIsMCx7InN0eWxlIjp7ImJvZHkiOnsibmFtZSI6ImRvdHRlZCJ9fX1dLFs1LDcsIlxcZXhpc3RzICEiLDIseyJzdHlsZSI6eyJib2R5Ijp7Im5hbWUiOiJkb3R0ZWQifX19XSxbMyw4XSxbMyw3LCIiLDAseyJjdXJ2ZSI6LTR9XSxbOCw3LCJcXGV4aXN0cyAhIiwwLHsic3R5bGUiOnsiYm9keSI6eyJuYW1lIjoiZG90dGVkIn19fV1d
		\[\begin{tikzcd}
			M && {S(M)} \\
			{} & {M\oplus N} & {S(M)\otimes_A S(N)} & {} & D \\
			N && {S(N)}
			\arrow[from=1-1, to=2-5]
			\arrow[from=3-1, to=2-5]
			\arrow[from=1-1, to=2-2]
			\arrow[from=3-1, to=2-2]
			\arrow[from=1-3, to=2-3]
			\arrow[from=3-3, to=2-3]
			\arrow["{\exists !}", dotted, from=1-3, to=2-5]
			\arrow["{\exists !}"', dotted, from=3-3, to=2-5]
			\arrow[from=2-2, to=2-3]
			\arrow["{\exists !}", dotted, from=2-3, to=2-5]
		\end{tikzcd}\]
		\item 
		\begin{enumerate}
			\item % https://q.uiver.app/#q=WzAsNyxbMCwwLCIwIl0sWzEsMCwiRSJdLFsyLDAsIkFeMyJdLFszLDAsIkEiXSxbNCwwLCIwIl0sWzMsMSwiYSJdLFsyLDEsImEoeCx5LHopIl0sWzAsMV0sWzEsMl0sWzIsMywiIFxccGhpIiwwLHsib2Zmc2V0IjotMX1dLFszLDRdLFszLDIsIlxccHNpIiwwLHsib2Zmc2V0IjotMX1dLFs1LDYsIiIsMCx7InN0eWxlIjp7InRhaWwiOnsibmFtZSI6Im1hcHMgdG8ifX19XV0=
			\[\begin{tikzcd}
				0 & E & {A^3} & A & 0 \\
				&& {a(x,y,z)} & a
				\arrow[from=1-1, to=1-2]
				\arrow[from=1-2, to=1-3]
				\arrow["{ \phi}", shift left, from=1-3, to=1-4]
				\arrow[from=1-4, to=1-5]
				\arrow["\psi", shift left, from=1-4, to=1-3]
				\arrow[maps to, from=2-4, to=2-3]
			\end{tikzcd}\]
			\item $S(E)[T]$=$S(E)\otimes_A A[T]$=$S(E\oplus A)$=$S(A^3)=A[P,Q,T]$.
			\item 由(a).iii.
			\item 若$\varphi:A[P,Q]\rightarrow S(E)$为同构. $A[P,Q]$与$S(E)$中的可逆元均只有$\mathbb{R}$, 故$\varphi(\mathbb{R})=\mathbb{R}$.并且$\mathrm{Aut}(\mathbb{R})=\{\mathrm{Id}\}$. 故该映射保持$\mathrm{R}$. 容易验证$A[P,Q], A[U,V,W]/(xU+yV+zW)$中方程$X^2+Y^2+Z^2=1$的解均在$A$中, 故$\varphi(A)=A$. 复合上$A$的自同构, 不妨设$\varphi$保持$A$中所有元素. 设$c=\varphi(P),c'=\varphi(Q)$. 设$c=c_0+c_1+\dots$, $c'=c'_0+c'_1+\dots$. 其中$c_i,c'_i$均为$i$次齐次式. 注意到$Ac_1+Ac'_1=(AU+AV+AW)/(xU+yV+zW)=E$. 但容易证明$E$不可被两个元素生成, 由此得出矛盾. 
		\end{enumerate}
	\end{enumerate}
\end{proof}
\end{enumerate}
\end{document}
